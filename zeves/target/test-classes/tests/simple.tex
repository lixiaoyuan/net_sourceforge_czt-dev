\begin{zsection}
   \SECTION simple \parents standard\_toolkit
\end{zsection}

Empty proof

\begin{zproof}[emptyProof]
\end{zproof}

By single words

\begin{zproof}[singleWord]
cases;
next;
conjunctive;
disjunctive;
prenex;
rearrange;
reduce;
rewrite;
simplify;
\end{zproof}

By multiple words

\begin{zproof}[multipleWords]
trivial rewrite;
trivial simplify;
prove by
  rewrite;
prove by
    reduce;     % new lines in the middle of the proof command. comments cannot appear between "by" and "reduce"!
equality substitute;
\end{zproof}

Per command kind. ApplyCommand

\begin{zproof}[applyWordComplex]
apply xyz to expression x + 1;
apply abc to predicate a \in b;
apply qwerty;
\end{zproof}

Case analysis command

\begin{zproof}[caseAnalysisCmd]
cases;
next;
split P;
\end{zproof}

Normalization command

\begin{zproof}[normCmd]
conjunctive;    % some comments involving ";"...
disjunctive;
rearrange;
with normalization rewrite;
\end{zproof}

Quantifiers command

\begin{zproof}[qntCmd]
prenex;
instantiate x == 10 + 5, y == \{ a: \nat; b:\nat | P(a,b) \},
    z == 0, x\$ == 10;  % special variables?
prenex;
\end{zproof}

Simplification command

\begin{zproof}[simpCmd]
prove by reduce;
rewrite;
reduce;
trivial simplify;
trivial rewrite;
simplify;
prove by rewrite;
% implicit prove by rewrite
prove;
\end{zproof}

Susbstitution command

\begin{zproof}[substCmd]
equality substitute;
equality substitute E;
invoke predicate P;
invoke Name;
invoke;
\end{zproof}

Use command

\begin{zproof}[useCmd]
use otherName;
use otherName2[X, Y];
use name[X, \power \nat][x := 10];
use name[x:= 10];
use name[X, \power \nat][x:= 10,
  y := \{ a: \nat; b: \nat \}];

% Funny case: Z/EVES doesn't like this newline between gen actuals and repl.
% and the trouble is that the ContextFreeScanner "eats" new lines here, so they
% don't appear in the token stream. This is necessary to avoid problem when
% tokenising things like "prove NL by NL rewrite" and "by NL" isn't a keyword.
% also, I would need to add loads of optional newlines in the parse too :-( TODO (NO HURRY).
use name[X, \power \nat]
[x:= 10,
  y := \{ a: \nat; b: \nat \}];
\end{zproof}

With commands

\begin{zproof}[withCmds]
with predicate (P) rewrite;
with expression (E) reduce;
with enabled (A, B, C) rewrite;
with disabled (A, B, C) reduce;
\end{zproof}

Combined commands

\begin{zproof}[combinedCmd1]
with normalization
      with
          normalization
              reduce
      ;           % and some comments

% very intricate combined command
with enabled (A, B, C)
  with normalization
    with predicate (P)
      prove by reduce;
\end{zproof}


Special commands: weird names and all...

\begin{zproof}[bla\$dom]
prenex;
apply inDom to predicate x = \dom~f \cup \dom~r;
apply S\$member;
instantiate x == 10 + 5, y == \{ a: \nat; b:\nat | P(a,b) \},
    z == 0, wierd$ == 20; % wierd case for variable name that is allowed...
prenex;
with enabled (S\$member, inDom) prove;
\end{zproof}

