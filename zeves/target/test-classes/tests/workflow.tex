\begin{document}
\begin{zsection}
  \SECTION workflow \parents standard\_toolkit
\end{zsection}

%%%%%%%%%%%%%%%%%%%%%%%%%%%%%%%%%%%%%%%%%%%%%%%%%%%%%%%%%%%%%%%%%%%%%%%%%%%%%%%%%%%%%%%%%%
\section{Operators and auxiliary types}

\newcommand{\readby}{\zrelop{readBy}}
\newcommand{\writes}{\zrelop{writes}}
\newcommand{\readableTo}{\zrelop{readableTo}}
\newcommand{\canWrite}{\zrelop{canWrite}}
\syndef{\readby}{inrel}{"readBy"}
\syndef{\writes}{inrel}{"writes"}
\syndef{\readableTo}{inrel}{"readableTo"}
\syndef{\canWrite}{inrel}{"canWrite"}

%%Zinword \readBy readBy
%%Zinword \writes writes
%%Zinword \readableTo readableTo
%%Zinword \canWrite canWrite

\begin{zed}
\relation (\varg \readBy \varg)
\\
\relation (\varg \writes \varg)
\\
\relation (\varg \readableTo \varg)
\\
\relation (\varg \canWrite \varg)
\end{zed}

\begin{zed}
   [Service, Data]
%\also
%   unclassified ~~==~~ 0
%\also
%   restricted ~~==~~ 1
%\also
%   classified ~~==~~ 2
%\also
%   confidential ~~==~~ 3
%\also
%   secret ~~==~~ 4
%\also
%   topsecret ~~==~~ 5
%\also
%   Level ~~==~~ \{~ unclassified, restricted, classified, confidential, secret, topsecret ~\}
\end{zed}

begin{axdef}
   d0, d1, d2, d3: Data
\\
   s0, s1, s2, s3: Service
end{axdef}

begin{theorem}{lServDatWitness}
   (\exists s: Service @ true) \land (\exists d: Data @ true)
end{theorem}

%%%%%%%%%%%%%%%%%%%%%%%%%%%%%%%%%%%%%%%%%%%%%%%%%%%%%%%%%%%%%%%%%%%%%%%%%%%%%%%%%%%%%%%%%%
\section{Workflow}

\begin{schema}{Workflow}
   \_ \readBy \_ : Data \rel Service
\\
   \_ \writes \_ : Service \rel Data
\\
   \_ \readableTo \_ : Data \rel Service
\\
   \_ \canWrite \_ : Service \rel Data
\\
   wf, graph: Data \rel Data
\where
   % A workflow is the transitive closure of the composition between
   % data read by services that write data subsequentially to it.
   wf = ((\_ \readBy \_) \comp (\_ \writes \_))\plus
\\
   graph = ((\_ \readableTo \_) \comp (\_ \canWrite \_))\plus
\end{schema}

\begin{schema}{WFAuthorised}
   Workflow
\where
   % What's actually read is within what's possible to read - AKA Authorisation
   % What's actually written is within wha't possible to write
   (\_ \readBy \_) \subseteq (\_ \readableTo \_)
 \\
   ((\_ \writes \_), (\_ \canWrite \_)) \in (\_ \subseteq \_)
\end{schema}

\begin{theorem}{frule fWFAuthorisedReadByWithinReableToDom}
    \forall Workflow @ \dom~(\_ \readBy \_) \in \power~(\dom~(\_\readableTo \_))
\end{theorem}
