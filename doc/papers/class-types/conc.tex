\section{Conclusions and Discussion}

In this paper, we discuss some of the difficulties encountered for
typechecking Object-Z, such as inheritance, polymorphism, implicit
downcasting, generic classes, and feature renaming. We present the
model for class types that is used in the CZT Object-Z
typechecker. This model of class types is flexible enough to represent
the types of class reference expressions, polymorphic expressions, and
class union expressions, all found in Object-Z, and also to handle
renaming and generic class instantiations.

We also present two type unification algorithms used for type
inferencing and detecting type inconsistencies between
expressions. The first enforces a weak form of unification
between two class types, and the second enforces a strong
form. The strong form of type consistency is used to detect possible
implicit downcasts, while the weak form permits implicit downcasts
provided there is at least one legal downcast. The CZT Object-Z
typechecker gives users the flexibility to choose which form of typing
to use.

In the typechecker, we have found that the model of class types
presented in this paper is sufficient for detecting type
inconsistencies between Object-Z expressions and it makes type
inference and applying the type rules straightforward. It also fits in
well with the existing Z type system.

\subsection{Other Work}

We have drafted a set of Object-Z type rules using a combination of
the calculus used by Palsberg \cite{palsberg95} and the type-inference
rule metalanguage defined in \theStandard \cite{isoz}, and will
continue to review and refine these. This will be coordinated with the
continued effort developing the open-source CZT project.
