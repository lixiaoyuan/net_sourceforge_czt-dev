\documentclass{article}
\usepackage{ltcadiz}
\begin{document}
\noindent

\begin{zsection}
  \SECTION shunting \parents oz\_toolkit, standard\_toolkit
\end{zsection}

First, we begin by defining the set of positions on the board. We define
the numbering on the board as if it were a square (6x7), but don't include
the positions not on the board. Hence define
\begin{zed}
  Posn == \{3,4,10,11\} \cup (15 \upto 28) \cup \{31,32,38,39\}
\end{zed}

We then define a relationship $next$, between cell positions, which
indicates the neighbours of cells.
\begin{zed}
 \relation (\varg next \varg)
\end{zed}

\begin{axdef}
  \varg next \varg : Posn \rel Posn
\where
  \forall m,n: Posn @ \\
    \t1 m~next~n \iff \\
       \t2 m - n \in \{ \negate 7, 7 \} \lor \\
       \t2 (m - n \in \{ \negate 1 , 1 \} \land m \div 7 = n \div 7)
\end{axdef}

Next, we define the class $Circle$, which represents a circle on the
board. To move the circle, a direction is entered. Hence, we define the
$DIR$, which represents directions.

\begin{zed}
  DIR ::= LEFT | RIGHT | UP | DOWN
\end{zed}

\begin{class}{Circle}
\also
\begin{state}
  posn: Posn
\end{state} \classbreak
\begin{op}{move}
  \Delta(posn) \\
  d? : DIR
\where
  posn~next~posn' \\
  d? = LEFT \implies posn' = posn - 1 \\
  d? = RIGHT \implies posn' = posn + 1 \\
  d? = UP \implies posn' = posn - 7 \\
  d? = DOWN \implies posn = posn + 7 \\
\end{op}
\end{class}

Now, we define a class $BlackCircle$ that extends the $Circle$ class to
represent a black circle by including a $count$ for the number of times a
circle is moved.

\begin{class}{BlackCircle}
\also
Circle\\
\begin{state}
  count: \nat
\end{state} \classbreak
\begin{init}
  count = 0
\end{init} \classbreak
\begin{op}{move}
  \Delta(count)
\where
  count' = count + 1
\end{op}
\end{class}

Now, we define the $Shunting$ class, which maintains the state of the
board.

\begin{class}{Shunting}
\also
\begin{state}
  black: BlackCircle \\
  white: \power Circle \\
\where
  \#white = 4 \\
  \forall m,n: white @ \\
      \t1 (m \neq n \iff m.posn \neq n.posn) \land \\
      \t1 m.posn \neq black.posn
\end{state} \classbreak
\begin{init}
  black.posn = 22 \\
  \{w: white @ w.posn\} = \{18,23,27,32\} \\ 
\end{init} \classbreak
\begin{op}{end}
  count! : \nat
\where
  \{w: white @ w.posn\} = \{17,18,24,25\} \\
  count! = black.count
\end{op}  \classbreak
move \sdef black.move \land\\
         \t2 ( \dgch u, t : white | t.posn = black'.posn \land
                u \neq t \land u.posn \neq t.posn @\\
		\t3 t.move\\
		\t2 \gch \\
		\t2 end )
\end{class}

\end{document}
